\documentclass{article}
\usepackage{mathtools}
\usepackage{amssymb}
\usepackage{xeCJK}
\usepackage{geometry}
\geometry{a4paper, tmargin = 1.5cm, bmargin = 1.5cm, lmargin = 1.5cm, rmargin=0cm}
\pagestyle{empty}
\setcounter{secnumdepth}{-2}
\setcounter{tocdepth}{-2}
\setlength\parindent{0pt}
\setlength\tabcolsep{0pt}
\setlength\lineskip{\jot}
\setlength\lineskiplimit{\lineskip}
\newcommand\K[2][.]{\ensuremath{%
	\left\{%
		\begin{tabular}{ll}#2\end{tabular}%
	\right#1^{\mathstrut}_{\mathstrut}%
}\null}
\renewcommand\P[2][p.\thinspace]{\textsuperscript{\rmfamily\mdseries\upshape[#1#2]}}
\newcommand\I[1]{\textcircled{{\scriptsize#1}}\thinspace\ignorespaces}
\newcommand\M[1]{\begin{tabular}[c]{l}#1\end{tabular}}

\begin{document}
\small
\section{毛泽东思想}

\subsection{是什么?}

关于中国革命和建设的正确的理论原则和经验总结;马克思列宁主义在中国运用与发展。——第一次历史性飞跃、第一大理论成果

\subsection{如何形成}

\K[\}]{
	萌芽\P{4}:国内革命战争时期:《中国社会各阶级的分析》——提出新民主主义革命的基本思想\\
	形成\P{4}:土地革命战争时期:《反对本本主义》等——阐述农村包围城市、武装夺政权思想(初步形成标志)\\
	成熟\P{5}:遵义会议到抗战胜利:《新民主主义论》等——系统阐述新民主主义革命理论(趋于成熟的标志)\underline{七大写入党章}\\
	继续发展\P{6}:解放战争时期和建国后:《论十大关系》等——人民民主专政理论、社会主义改造理论、两类矛盾学说
}\M{问题导向\\经验总结}

\subsection{主要内容}

\K{
	革命\K{
		新民主主义革命\K[\}]{
			形成根据\P{19}:\I1近代中国国情、社会性质、主要矛盾;\I2旧革命失败;\I3新革命曲折\\
			总路线:革命的对象\P{25} 动力\P{26} 领导力量\P{29} 革命性质与前途\P{29}\\
			道路和经验:\I1农村包围城市、武装夺取政权\P{35};\\
			      \I2三大法宝:统一战线、武装斗争、党的建设\P{36}
		}新中国成立\\
		社会主义革命\K[\}]{
			过渡性社会:从新中国成立到社会主义改造基本完成——新民主主义社会\\
			过渡时期的总路线\P{56}:一化三改造\P{48}$\implies$两个转变(建设与改造并举)\P{56}\\
			社会主义改造道路\P{51}:农业、手工业、资本主义工商业$\implies$和平方式\P{57}
		}社会主义制度建立\P{59}
	}\\
	建设\K[\}]{
		理论\K[\}]{
			调动一切积极的因素为社会主义事业服务\P{65}——《论十大关系》:探索中国社会主义建设道路\\
			正确处理社会主义矛盾的思想:社会主要矛盾、敌我矛盾、人民内部矛盾\P{68}\\
			走中国工业化道路的思想:\P{73}:有别于苏联;农轻重比例问题
		}\\
		成就:\I1巩固发展了社会主义制度;\I2奠定物质基础;\I3理论准备与宝贵经验\P{77}\\
		教训:六大经验教训——要实事求是;抓主要矛盾;有步骤有秩序;民主法制;党的建设;对外开放\P{78}
	}\M{有中国特点的\\社会主义建设道路\\探索}
}

\subsection{历史地位}

\K[\}]{
	马克思主义中国化的第一个重大理论成果\P{14}\\
	中国革命和建设的科学指南\P{15}\\
	中国共产党和中国人民宝贵的精神财富\P{16}
}活的灵魂\P{11}\K[\}]{
	实事求是\\
	群众路线\\
	独立自主
}\M{
	久经磨难的中华民族从此站起来了\\
	为当代中国一切发展进步奠定根本政治前提和制度基础
}

\section{邓小平理论}

\subsection{形成过程}

\K[\}]{
	\I11978年:思想路线、组织路线、政治路线的拨乱反正\\
	\I2十二大:提出建设没有中国特色的社会主义\\
	\I3十三大:系统论述初级阶段理论等,标志轮廓形成\\
	\I4十四大:初步回答什么是社会主义、怎样建设社会主义
}\M{十五大正式提出“邓小平理论”\\写入党章。1999年载入宪法。}\\

\subsection{内容体系}
\K[\}]{
	哲学基础:解放思想实事求是的思想路线(邓小平理论活的灵魂、精髓)\\
	理论主题\K{
		本质\P{95}:解、发、消、消、最——社会主义根本任务\P{102}:发展生产力\\
		道路\P{100}:党在初级阶段的基本路线:一个中心、两个基本点(一百年不动摇)
	}\\
	立论根据\K{
		时代主题\P{87}:和平与发展——当今世界是开放的世界,中国的发展离不开世界\\
		发展方位\P{98}:社会主义初级阶段——当代中国最大“实际”,基本国情(已进入但不发达)\\
		主体条件:三落三起的人声、个人意志与品格
	}\\
	系统论证\K{
		经济\K{
			三步走\P{103}:温饱、小康、中等发达国家水平(台阶式发展思想、共同富裕思想)\\
			改革开放\P{105}:社会主义制度的自我完善和发展(中国的第二次革命、三个有利于标准)\\
			市场经济\P{107}:计划和市场不是划分社会主义的标志
		}\\
		社会\P{108}:两手抓,两手都要硬\K{物质文明与精神文明\\建设与法制\\改革开放与惩治腐败}\\[-0.5em]
		统一\P{110}:一国两制、和平统一\\
		国际\P{106}:独立自主和平外交,实行对外开放的基本国策\\
		党建\P{112}:中国问题的关键在党——坚持、加强和改善党的领导(党的制度建设)
	}
}\M{基本理论问题:\\什么是社会主义、\\怎样建设社会主义\P{94}}\\

\subsection{历史地位}

\K[\}]{
	马克思列宁主义、毛泽东思想的继承和发展\P{114}\\
	中国特色社会主义理论体系的开篇之作\P{115} 响亮提出走自己的路\\
	改革开放和社会主义现代化建设的科学指南\P{116}
}\M{
	马克思主义基本原理与当代中国实际与时代特征相结合\\
	邓小平是我国改革开放和社会主义现代建设的总设计师
}

\section{习近平新思想}

\subsection{理论框架\textmd{(八个明确 十四个坚持)}}

\K{
	主题\P[]{第八章}:坚持和发展中国特色社会主义\P{183}(核心要义)——确立四个自信\K[\}]{道路自信\\理论自信\\制度自信\\文化自信\hspace*{7em}}伟大事业\\
	总任务\P[]{第九章}:实现社会主义现代化和中华民族伟大复兴\K[\}]{
		中国梦的本质:国家富强,民族振兴,人民幸福\P{197}\\
		现代化强国的战略安排:“两步走”\P{203}
	}伟大梦想\\
	\hspace*{-\nulldelimiterspace}$\left.\M{
		总体布局\P[]{第十章}\K{
			经济:建设现代化经济体系&\P{207}(新理念、供给侧、六大着力点)\\
			政治:发展社会主义民主政治&\P{214}(发展道路、制度体系、统一战线、一国两制)\\
			文化:繁荣发展社会主义文化&\P{223}(意识形态工作、核心价值观、文化强国)\\
			社会:发展中保障和改善民生&\P{231}(教育、医疗、卫生、就业、脱贫——社会治理创新)\\
			生态:建设美丽中国&\P{237}(生态文明的核心;生态文明体制改革)
		}\\
		战略布局\P[]{第十一章}\K{
			全面建成小康社会:实现第一个百年目标\P{244}&——三大攻坚战\\
			全面深化改革:国家治理体系和治理能力现代化\P{249}&——处理好五大关系\\
			全面依法治国:中国特色社会主义法治道路\P{255}&——党的领导是最根本保障\\
			全面从严治党:四大考验、四大危险、五大建设\P{260}&——政治建设是根本性建设
		}\\
		\rule{0pt}{8em}
	}\right\}$\M{伟大斗争\\(总体国家安全观)\P{235}}\\[-8em]
	系统保障\K{
		内部保障\P[]{第十二章}\K{
			“一国两制”、统一祖国\P{219}——三大历史任务之一\\
			全面推进国防和军队现代化\K{
				中国特色强军之路\P{268}\\
				军民融合深度发展\P{277}
			}
		}\\
		外部保障\P[]{第十三章}\K[\}]{
			独立自主和平外交:建立新型国际关系\P{285}\\
			构建人类命运共同体:“一带一路”\P{289}
		}中国特色大国外交\\
		根本保证\P[]{第十四章}\K[\}]{
			民族复兴关键在党(最本质特征、最大优势)\P{295}\\
			坚持党对一切工作的领到(大党执政大国的本领)\P{298}\hspace*{10em}
		}\M{伟大工程\\(起决定性作用)}
	}
}

\subsection{客观根据\P[]{第八章}}

\K[\}]{
	新矛盾:人民日益增长的美好生活需要和不平衡不充分发展之间的矛盾\P{178}&——必然性\\
	新方位:中国特色社会主义进入了新时代(5个是,3个意味着)\P{180}&——必然性\\
	新变革:全方位开创性的成就;深层次根本性\P{175}&——可能性\\
	新征程:两个一百年奋斗目标(建党100年、建国100年)\P{203}&——必要性
}\M{新时代之中国\\(百年未有之大变局)}

\subsection{主体条件}

\K[\}]{
	(1)战略定力和战略思维的统一:统筹全局,重点突破,防止颠覆性错误。\\
	(2)问题导向和价值导向的结合:问题是时代声音,人心是最大政治。\\
	(3)历史意识和世界意识的相融:时间坐标与空间坐标交汇。\\
	(4)理论思维和辩证思维的贯通:非经验思维;马克思主义理论思维的优势。\\
	(5)创新思维和底线思维的协调:守成出新,着眼最坏处,立足最低点。
}辩证唯物主义和历史唯物主义

\subsection{历史地位\P[]{第八章}}

\K[\}]{
	(1)马克思主义中国化的最新成果:中特理论体系组成部分/当代中国马克思主义\P{189}\\
	(2)新时代的精神旗帜:举旗定向——走什么路(四个自信、四个伟大)\P{190}\\
	(3)实现中华民族伟大复兴的行动指南——治党治国治军的基本遵循\P{191}
}\M{
	党的使命\\
	国家前途\\
	人民福祉\\
	民族命运
}

\end{document}
