\chapter{波动光学}

\section{光的本质}

光波是电磁波

同一媒质中的相对光强:$I={E_0}^2$

\section{光的相干性}

\subsection{发光机制}

\subsubsection{光源}

\begin{enumerate}
    \item 普通光源
          \begin{enumerate}
              \item 热光源:热能激发原子能级跃迁
              \item 冷光源:化学能,电能等激发
          \end{enumerate}
    \item 激光光源
\end{enumerate}

原子发光特点:
\begin{enumerate}
    \item 随机性
    \item 间歇性
    \item 各原子各级发光独立,波列互不相干
    \item 不相干性(独立光源不可能是一对相干光源:原子发光间歇而随机,振动方向和相位差不可能相同)
\end{enumerate}

\subsection{相干光源}

相干光源条件:
\begin{enumerate}
    \item 振动\textbf{频率}相同
    \item 振动\textbf{方向}相同
    \item \textbf{相位差}恒定
\end{enumerate}

原子自发辐射的间断性和相位随机性,不利于实现干涉条件。

\begin{equation}
    x_1+x_2=\sqrt{A_1^2+A_2^2+2A_1A_2\cos{(\varphi{}_2-\varphi{}_1})}\cos{(\omega{}t+\varphi{})}
\end{equation}

相长、相消:
\begin{equation}
    \begin{aligned}
        \delta{} & =r_2-r_1=\pm{}k\lambda      \\
        \delta{} & =r_2-r_1=\pm{}(2k+1)\lambda
    \end{aligned}
\end{equation}

\subsection{波动几何描述}

波线:
波面:

平面波:
球面波:

\section{惠更斯原理}

惠更斯原理:媒质中波动到的各点,都可以看作新波源,子波的包络面就是该时刻的波面。

\subsection{相干光的获得}

干涉光的获得:
\begin{enumerate}
    \item 分波面法
    \item 分振幅法
\end{enumerate}

\section{杨氏双缝实验}

\subsection{明暗条纹位置的推导}

\subsubsection{明纹条件}

\begin{equation}
    \begin{aligned}
        \delta{}  & =r_2-r_1=d\sin{\theta}\approx{d\tan{\theta}}                          \\
                  & =\frac{xd}{D}=k\lambda{}                                              \\
        x         & =k\frac{D\lambda}{d}                         & k=0,\pm{1},\pm{2}\dots \\
        \Delta{x} & =\frac{D\lambda}{d}
    \end{aligned}
\end{equation}

\subsubsection{暗纹条件}

\begin{equation}
    \begin{aligned}
        \mu{}\pm\delta{} & =\frac{xd}{D}=(2k+1)\frac{\lambda{}}{2}                \\
        x                & =(2k+1)\frac{D\lambda}{2d}              & k=0,1,2\dots \\
        \Delta{x}        & =\frac{D\lambda}{d}
    \end{aligned}
\end{equation}
