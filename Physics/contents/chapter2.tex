\chapter{热力学}

\section{热力学过程}

系统从\textbf{平衡态}到另一\textbf{平衡态}的过程

准静止状态:\textbf{无限缓慢},\textbf{每个}中间态都可视为\textbf{平衡态}

\section{p-V 图}

\begin{enumerate}
    \item 点:一个平衡态
    \item 线:一个准静态过程
\end{enumerate}

\section{系统内能}

\begin{enumerate}
    \item 功(过程量)

          p-V 图与曲线对 p-V 轴积分所成面积即为功

          \begin{equation}
              dW=Fdl=pSdl
          \end{equation}

          \begin{equation}
              W=\int_{V1}^{V2}{pdV}
          \end{equation}

          \begin{enumerate}
              \item $W>0~$系统对外界作正功
              \item $W<0~$系统对外界作负功
          \end{enumerate}

    \item 热(过程量)

          \begin{enumerate}
              \item  同:
                    \begin{enumerate}
                        \item 过程量:与过程有关
                        \item 等效性:对系统热状态改变的作用相同
                    \end{enumerate}

              \item 异:
                    \begin{enumerate}
                        \item 功:宏观运动-分子热运动
                        \item 功:分子热运动-分子热运动
                    \end{enumerate}
          \end{enumerate}

    \item 内能
          $E_2-E_1=W+Q~~~W+Q只与始末状态有关,与过程无关$

\end{enumerate}

\section{热力学第一定律}

系统吸收的能量,一部分使内能增加,另一部分用于系统对外作功

\begin{equation}
    Q=E_2-E_1+W=\Delta{}E+W
\end{equation}

\begin{equation}
    dQ=dE+dW
\end{equation}

\begin{equation}
    Q=\Delta{E}+\int_{V1}^{V2}{pdV}
\end{equation}

\begin{equation}
    C_V=\frac{i}{2}R
\end{equation}

\begin{enumerate}

    \item 等容过程

          \begin{equation}
              \frac{p_1}{T_1}=\frac{p_2}{T_2}=\mathit{const}
          \end{equation}

          \begin{equation}
              \nu{}=\frac{M}{M_\mathrm{mol}=\frac{pV}{RT}}
          \end{equation}

          \begin{equation}
              Q_V=E_2-E_1=\nu{}\frac{i}{2}R\Delta{T}
          \end{equation}

          p-V 图为横线

          系统从外界吸收的热量全部转化为内能的增加

          定容摩尔热容$C_V$:1mol 理想气体在等体过程中,温度变化 1 摄氏度所变化的热量

    \item 等压过程

          \begin{equation}
              \frac{V_1}{T_1}=\frac{V_2}{T_2}=\mathit{const}
          \end{equation}


          \begin{equation}
              \Delta{}E=E_2-E_1=\nu{}C_V\Delta{T}
          \end{equation}

          \begin{equation}
              Q_p=\nu{}C_V\Delta{T}+\nu{}R\Delta{T}
          \end{equation}

          |

          定压摩尔热容$C_p$:1mol 理想气体在等压过程中,温度变化 1 摄氏度所变化的热量

          \begin{equation}
              C_p=C_V+R=\frac{i+2}{2}R
          \end{equation}

          比热容比:

          \begin{equation}
              \gamma{}=\frac{C_p}{C_V}=\frac{i+2}{i}
          \end{equation}

    \item 等温过程

          \begin{equation}
              Q_T=W=\int_{V_1}^{V_2}{\frac{m}{M}\frac{Rt}{V}dV}=\frac{m}{M}RT\ln\frac{V_2}{V_1}=\frac{m}{M}RT~\ln{\frac{p_1}{p_2}}
          \end{equation}

          p-V 图为曲线

          \begin{enumerate}
              \item 等温膨胀吸热作功
              \item 等温压缩放热被作功
          \end{enumerate}

    \item 绝热过程

          系统对外界作功,通过系统内能减小完成

          热一律:
          \begin{equation}
              dW+dE=0\\
              dQ=0
          \end{equation}

          \begin{equation}
              \Delta{E}=\frac{m}{\mu}C_V(T_2-T_1)\\
              \Delta{W}=-\frac{m}{\mu}C_V(T_2-T_1)\\
              Q=0
          \end{equation}

          \begin{equation}
              \begin{aligned}
                  V^{\gamma{}-1}T=\mathit{const} \\
                  pV^{\gamma}=\mathit{const}     \\
                  p^{\gamma{}-1}T^{-\gamma}=\mathit{const}
              \end{aligned}
          \end{equation}

          \begin{equation}
              \gamma{}=\frac{C_p}{C_V}=\frac{i+2}{i}
          \end{equation}

          绝热膨胀 $T_1>T_2,W>0$
          绝热压缩 $T_1<T_2,W<0$

          绝热过程曲线斜率\textbf{大于等于}等温过程

\end{enumerate}

\section{循环过程}

热机:持续将热量转变为功的机器

工质:吸收热量,对外作功

p-V 图呈闭合曲线

\begin{enumerate}
    \item 顺时针:正循环,热机
    \item 逆时针:负循环,制冷机
\end{enumerate}

\subsection{正循环}

\begin{equation}
    \Delta{W}=W_1+W_2>0
\end{equation}

闭合曲线包裹过程为净功

热机效率

\begin{equation}
    \eta=\frac{\Delta{W}}{Q_{in}}=\frac{Q_{in}-Q_{out}}{Q_{in}}=1-\frac{Q_{out}}{Q_{in}}
\end{equation}

绝热线、等温线不能相交

\section{热力学第二定律}

