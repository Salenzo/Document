\chapter{气体动理论}
\newpage

\section{热力学系统}

\begin{table}[!htbp]
    \centering
    \begin{tabular}{l|c|c}
        \toprule
                 & 能量交换 & 物质交换 \\
        \midrule
        孤立系统 & false    & false    \\
        封闭系统 & true     & false    \\
        开放系统 & true     & true     \\
        \bottomrule
    \end{tabular}
\end{table}

\section{平衡态}

\begin{enumerate}
    \item 单一性
    \item 稳定性
    \item 热动平衡
\end{enumerate}

\section{理想气体物态方程}

$1\,\mathrm{atm}=1.013\times10^5$

$mathrm{Pa}=760mmHg\\T=t+273.15$

\begin{enumerate}
    \item 波义耳定律(T)
          $p_1V_1=p_2V_2$
    \item 盖$\cdot$吕萨克定律(P)
          $\frac{V_1}{T_1}=\frac{V_2}{T_2}$
    \item 查理定律(V)
          $\frac{p_1}{T_2}=\frac{p_2}{T_2}$
\end{enumerate}

理想气体物态方程:

\begin{equation}
    pV=\frac{m'}{\mu}RT\hspace{10ex}m'=Nm ,~ \mu=N_Am
\end{equation}

理想气体压强公式:

\begin{equation}
    \overline{v^2}=\frac{v_1^2+\ldots+V_n^2}{N}=\frac{1}{N}\sum_{i=1}^{N}{v_i^2}
\end{equation}

$pV=\frac{m}{M_\mathrm{mol}}RT=\nu{}RT$

$m:气体质量(kg)\\M_\mathrm{mol}:气体摩尔质量\\R:气体普适常量\\\nu:摩尔数$

理想气体常数:

$p(atm),V(L),T(K)\Rightarrow R=8.2\times 10^{-2}atm\cdot L/(mol\cdot K)\\p(atm),V(m^3),T(K)\Rightarrow R=8.31 J/(mol\cdot K)$

玻尔兹曼常数: $k=\frac{R}{N_A}$

$p=nkT \\ p=\frac{2}{3}n \overline{\varepsilon_k}\\\overline{\varepsilon_k}=\frac{1}{2}m\overline{v}^2=\frac{3}{2}kT$

\section{能量均分定理}

\begin{table}[!htbp]
    \centering
    \begin{tabular}{l|c}
        \toprule
                 & 自由度($\frac{1}{2}kT$/自由度) \\
        \midrule
        质点     & i=3                            \\
        刚体     & i=6                            \\
        刚性分子 & i=t+r                          \\
        \bottomrule
    \end{tabular}
\end{table}

\section{内能}

\begin{equation}
    E=N_A\overline{\varepsilon{}}=N_A\frac{i}{2}kT\implies{E}=\frac{i}{2}RT
\end{equation}

\section{麦克斯韦速率分布律}

\begin{enumerate}
    \item \textbf{单个}分子速率分布具有\textbf{偶然性}
    \item \textbf{大量}分子速率分布具有\textbf{规律性}
\end{enumerate}

麦克斯韦分布函数:表示\textbf{单位速率区间的分子数占总数的百分比}

\begin{equation}
    f(v)=\frac{1}{N}\frac{dN}{dv}
\end{equation}

\begin{equation}
    f(v)=4\pi \left(\frac{m_0}{2\pi kT}\right)^{3/2}e^{-m_0v^2/2kT}v^2
\end{equation}

\section{三种统计速率}

\begin{enumerate}

    \item 最概然速率
          \begin{equation}
              v_p=\sqrt{\frac{2kT}{m}}\approx 1.41\sqrt{\frac{RT}{M}},\approx 1.41\sqrt{\frac{kT}{m}}
          \end{equation}

    \item 平均速率
          \begin{equation}
              \overline {v}=\frac{1}{N}\sum_{i=1}^n{v_iN_i}=\sqrt{\frac{8kT}{\pi m}}\approx 1.60\sqrt{\frac{RT}{M}},\approx 1.60\sqrt{\frac{kT}{m}}
          \end{equation}

    \item 方均根速率$\sqrt{\overline{v}^2}$
          \begin{equation}
              \overline{v}^2=\frac{1}{N}\sum_{i=1}^n{v_i^2N_i},\sqrt{\overline{v}^2}=\sqrt{\frac{3kT}{m}}\approx 1.73\sqrt{\frac{RT}{M}},\approx 1.73\sqrt{\frac{kT}{m}}
          \end{equation}

\end{enumerate}

比较:$v_p<\overline{v}<\sqrt{\overline{v}^2}$

归一化条件:
\begin{equation}
    \begin{aligned}
        & \int^{0}_{\infty}{f(v)dv}=1 \\
        & dS=f(v)dv=\frac{dN}{N}
    \end{aligned}
\end{equation}

\section{平均自由程}

单位时间内平均碰撞次数:$~\overline{Z}=\sqrt{2}\pi{}d^2vn$

平均自由程\textbf{每两次}碰撞之间,一个分子自由运动的\textbf{平均路程}。

\begin{equation}
    \overline{\lambda}=\frac{kT}{\sqrt{2}\pi{}d^2p}\qquad\overline{\lambda{}}\propto\frac{1}{p},T\qquad d=10^{-10}m
\end{equation}

\newpage