\chapter{同步时序电路}
\newpage

\section{结构}

\begin{enumerate}

    \item 组合电路
    \item 储存电路

\end{enumerate}

\begin{enumerate}

    \item 输出函数

          \begin{equation}
              Z_i=f_i(x_1,\dots,x_n,y_1,\dots,y_r),~i=1,\dots,m
          \end{equation}

    \item 激励函数

          \begin{equation}
              Y_j=g_j(x_1,\dots,x_n,y_1,\dots,y_r),~j=1,\dots,r
          \end{equation}

          统一的时钟信号(不能太短)来临后,电路状态才改变,且只有一次。

\end{enumerate}

\newpage

时钟信号起同步作用

\begin{table}[!htbp]
    \centering
    \begin{tabular}{lccc}
        \toprule
        信号来临前 & 现态 & $y^{n}$   \\
        \midrule
        信号来临后 & 次态 & $y^{n+1}$ \\
        \bottomrule
    \end{tabular}
\end{table}

\section{描述}

\begin{enumerate}
    \item 状态表
    \item 状态图
\end{enumerate}

\newpage

\section{触发器}

\begin{enumerate}
    \item 储存电路
    \item 能储存一位二进制数
    \item 在任一时刻只处于一种\textbf{稳态}
\end{enumerate}

\subsection{R-S 触发器}

\begin{enumerate}
    \item 基本型(锁存器)
    \item 直接 复位-置位
    \item 组成:
          交叉耦合或非门
          交叉耦合与非门
    \item 时钟型
          两个控制与非门+两个基本耦合与非门
\end{enumerate}

空翻现象:由于时钟信号宽度而多次翻转,可由主从触发器(串联)解决。

\newpage

\subsection{D 触发器}

为了解决 R-S 触发器输入同为 1 时触发器状态不确定问题。

\begin{enumerate}
    \item 单输入端
    \item 输入信号转换为互补信号
\end{enumerate}

\subsection{J-K 触发器}

为了解决 R-S 触发器输入同为 1 时触发器状态不确定问题,同时使触发器有两个输入端。

\subsection{T 触发器}

JK 端合并为 T 端。

\newpage

\section{电路分析}

\begin{enumerate}
    \item 根据电路,列出输出函数表达式和激励函数表达式
    \item 建立状态转移真值表
    \item 作出电路状态表,画出状态图
    \item 用文字和时间图描述电路逻辑功能
\end{enumerate}

\section{电路设计}

\begin{enumerate}
    \item 根据逻辑要求,作出原始状态图和状态表
    \item 状态简化
    \item 状态编码
    \item 求出激励函数和输出函数表达式
    \item 画出逻辑电路图
\end{enumerate}

\newpage
