\chapter{组合逻辑电路}
\newpage

\section{逻辑门}

\begin{enumerate}

    \item 与门

          符号:$^A_B\!\!=\!\!\boxed{\&}\!\!-\!\!{\scriptstyle F}$

          表达式:$F=A \cdot B$

    \item 或门

          符号:$^A_B\!\!=\!\!\boxed{\geq{1}}\!\!-\!\!{\scriptstyle F}$

          表达式:$F=A+B$

    \item 非门

          符号:$A\!\!-\!\!\boxed{1}{\scriptstyle\circ}\!\!-\!\!{\scriptstyle F}$

          表达式:$F=\overline A$

    \item 复合逻辑门

          \begin{enumerate}

              \item 与非门

                    符号:$^A_B\!\!=\!\!\boxed{\&}{\scriptstyle\circ} \!\!-\!\!{\scriptstyle F}$

                    表达式:$F=\overline{AB}$

              \item 或非门

                    符号:$^A_B\!\!=\!\!\boxed{\geq{1}}{\scriptstyle\circ} \!\!-\!\!{\scriptstyle F}$

                    表达式:$F=\overline{A+B}$

              \item 与或非门

                    表达式:$F=\overline{AB+CD}$

              \item 异或门

                    符号:$^A_B\!\!=\!\!\boxed{={1}} \!\!-\!\!{\scriptstyle F}$

                    表达式:$F=A\oplus{}B=A\overline{B}+\overline{A}B$

                    同或运算:$F=A\odot{}B=AB+\overline{AB}$

          \end{enumerate}

\end{enumerate}

\newpage

\begin{enumerate}

    \item 根据输入逐级写出输出
    \item 化简逻辑功能
    \item 列出真值表
    \item 讨论功能

\end{enumerate}

\section{逻辑函数实现}

\begin{equation}
    \begin{aligned}
        F(A,B,C)=AB+\overline{A}C                                             \\
        =(\overline{A}+B)(A+C)                                                \\
        =\overline{\left[\overline{(A+C)}+\overline{(\overline{A}+B)}\right]} \\
        =\overline{(A\overline{B}+\overline{A}~\overline{})}
    \end{aligned}
\end{equation}

\newpage

\begin{enumerate}

    \item 与非

          \begin{enumerate}

              \item 化为最简**与或**式
              \item 变换为**与非\*2** 式

          \end{enumerate}

    \item 或非

          \begin{enumerate}

              \item 化为最简**或与**式
              \item 变换为**或非\*2**式

          \end{enumerate}


    \item 与或非

          \begin{enumerate}

              \item 化为最简**与或**式
              \item 变换为**与或非**式

          \end{enumerate}

    \item 异或(部分才能实现,但简单)

\end{enumerate}

\newpage

\section{组合逻辑电路分析}

\begin{enumerate}

    \item 输入标字母
    \item 从输入端按深度一层层写出逻辑函数
    \item 用前一层输出代入后一层并继续重复
    \item 简化逻辑函数,判断合理性
    \item 列出逻辑电路的真值表
    \item 判断功能并评价完善

\end{enumerate}

\section{组合逻辑电路设计}

\begin{enumerate}

    \item 根据逻辑要求构建真值表
    \item 根据真值表写出逻辑函数
    \item 将逻辑函数化简并转换成适当形式

\end{enumerate}

\newpage

\section{竞争与冒险}

原因:信号传输延迟。

用输入、输出时序图表示。

\subsection{竞争}

竞争:输入信号通过不同途径达到输出端的时间不同(随机过程)。

\begin{enumerate}

    \item 非临界竞争:不会产生错误
    \item 临界竞争:导致逻辑错误

\end{enumerate}

\subsection{冒险}

冒险(\textbf{暂时}、\textbf{瞬态}现象):输出端的尖脉冲。

\begin{table}[!htbp]
    \centering
    \begin{tabular}{lccc}
        \toprule
        {}                      & OUT\_EXPECTED & OUT\_ERROR & {}      \\
        \midrule
        \multirow{2}*{静态冒险} & 偏 1          & 1          & 1-0-1   \\
        \cline{2-4}
                                & 偏 0          & 0          & 0-1-0   \\
        \hline
        \multirow{2}*{动态冒险} & 偏 1          & 0-1        & 0-1-0-1 \\
        \cline{2-4}
                                & 偏 0          & 1-0        & 1-0-1-0 \\
        \bottomrule
    \end{tabular}
\end{table}

\newpage

\begin{enumerate}

    \item 代数判别法

          从\textbf{函数表达式结构}判别

          \begin{enumerate}

              \item 如果某变量同时以原变量反变量形式存在
              \item 将其他变量可能的取值代入
              \item 如果出现$x+\overline{x}$或$x\overline{x}$则可能产生冒险

          \end{enumerate}

    \item 卡诺图判别法

          \begin{enumerate}

              \item 画出各**与项**对应卡诺圈
              \item 如果两卡诺圈**相切**(存在共用的相邻最小项)则可能产生冒险
          \end{enumerate}

\end{enumerate}

\newpage
