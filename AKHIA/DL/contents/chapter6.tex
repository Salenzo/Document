\chapter{规模集成电路逻辑设计}
\newpage

\section{二进制并行加法器}

\subsection{逻辑功能}

产生两个二进制算术和。

\subsection{结构}

由全加器构成进位链无法完成\textbf{超前进位}。

74283

\newpage
\section{数值比较器}

\subsection{逻辑功能}

比较两个正数而确定其相对大小。

\subsection{结构}

7485

\newpage
\section{译码器}

\subsection{逻辑功能}

将 n 个输入变量变换为 $2^n$ 个输出函数,每个输出对应一个最小项。

\subsection{结构}

74138

\subsection{原理}

输入所有变量,展开函数到最小项,最小项化为输出的反。

\begin{equation}
    \begin{aligned}
        Y & = \Sigma_m{0,2,4,6,7}
          & =m_7+m_6+m_4+m_2+m_0                                                         \\
          & = \overline{Y_7}+\overline{Y_6}+\overline{Y_4}+\overline{Y_2}+\overline{Y_0} \\
          & = \overline{Y_0Y_2Y_4Y_6Y_7}
    \end{aligned}
\end{equation}


\newpage

\section{多路选择器}

\subsection{逻辑功能}

多路输入,单路输出,从多个输入中选择一个信号输出。

一个多路选择器只能实现一个逻辑函数。

\subsection{结构}

74153

\subsection{原理}

使能端对应各变量,输入端和卡诺图中一一对应。


\newpage
\section{计数器}

\subsection{逻辑功能}

对输入脉冲信号进行计数。

\subsection{结构}

分为同步,异步。

74193

\subsection{原理}


\newpage
\section{寄存器}

\subsection{逻辑功能}

存放数据或运算结果,具有接收数据、储存数据或传入数据的功能。

\subsection{结构}

74194

\subsection{原理}


\newpage
\section{只读存储器(ROM)}

\subsection{逻辑功能}

只读不写的存储器,断电后依然保存数据。

\subsection{结构}

\subsection{原理}



\newpage
\section{可编程逻辑阵列(PLA)}

\subsection{逻辑功能}

解决 ROM 存在地址译码和储存单元必须一一对应而浪费空间的缺陷,“与”阵列、“或”阵列都可以编程。

\subsection{结构}

\subsection{原理}


\newpage
\section{可编程阵列逻辑(PAL)}

\subsection{逻辑功能}

可编程逻辑阵列(PLA)的“与”阵列可编程,而“或”阵列固定。

\subsection{结构}

\subsection{原理}


\newpage
\section{通用阵列逻辑(GAL)}

\subsection{功能}

与 PAL 类似,但不采用熔丝 I/O,而是使用 OLMC,可以重复改写。

改进:
\begin{enumerate}
    \item 采用 $E^2CMOS$ 功耗低,速度快,可以电擦写和反复编程
    \item 输出结构配置了输出逻辑宏单元,可以编程选择输出组态
    \item 有加密单元,防复制,增加保密性
\end{enumerate}

\subsection{结构}

\subsection{原理}


\newpage
\section{高密度可编程逻辑器件(HDPLD)}

\subsection{功能}

EPLD,CPLD,FPGA

\subsection{结构}

\subsection{原理}



\newpage
