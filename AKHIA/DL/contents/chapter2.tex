\chapter{逻辑代数}
\newpage

\section{逻辑代数运算法则}

\subsection{逻辑运算}

\begin{enumerate}

    \item $F=A+B~~~(F=A\wedge B)$

          \begin{table}[!htbp]
              \centering
              \begin{tabular}{|c|c|c|}
                  \hline
                  A & B & F \\
                  \hline
                  0 & 0 & 0 \\
                  \hline
                  0 & 1 & 1 \\
                  \hline
                  1 & 0 & 1 \\
                  \hline
                  1 & 1 & 1 \\
                  \hline
              \end{tabular}
          \end{table}

    \item $F=A\cdot B~~~(F=A\vee B)$

          \begin{table}[!htbp]
              \centering
              \begin{tabular}{|c|c|c|}
                  \hline
                  A & B & F \\
                  \hline
                  0 & 0 & 0 \\
                  \hline
                  0 & 1 & 0 \\
                  \hline
                  1 & 0 & 0 \\
                  \hline
                  1 & 1 & 1 \\
                  \hline
              \end{tabular}
          \end{table}

    \item $F=\overline A$

          \begin{table}[!htbp]
              \centering
              \begin{tabular}{|c|c|}
                  \hline
                  A & F \\
                  \hline
                  0 & 1 \\
                  \hline
                  1 & 0 \\
                  \hline
              \end{tabular}
          \end{table}

\end{enumerate}

\newpage

\subsection{基本定律}

\begin{table}[!htbp]
    \centering
    \begin{tabular}{c|c|c}
        \toprule
        1 & $A+B=B+A$                                      & $AB=BA$                                 \\
        2 & $A+(B+C)=(A+B)+C$                              & $A(BC)=(AB)C$                           \\
        3 & $A+(BC)=(A+B)(A+C)$                            & $A(B+C)=AB+AC$                          \\
        4 & $A+0=A , A+1=0$                                & $A \cdot 1=A , A \cdot 0=0$             \\
        5 & $A+\overline A =1$                             & $A \cdot \overline A  =0$               \\
        6 & $A+A=A$                                        & $A \cdot A=A$                           \\
        7 &                                                & $\overline{\overline A}=A$              \\
        8 & $\overline{A+B}=\overline A \cdot \overline B$ & $\overline{AB}=\overline A+\overline B$ \\
        \bottomrule
    \end{tabular}
\end{table}

\newpage

\subsection{代入规则}

\subsection{反演规则}

\begin{equation}
    F\left\{
    \begin{aligned}
        1 & \Longleftrightarrow 0           \\
        + & \Longleftrightarrow \bullet     \\
        A & \Longleftrightarrow \overline A
    \end{aligned}
    \right\}\overline F
\end{equation}

\subsection{对偶规则}

\begin{equation}
    F\left\{
    \begin{aligned}
        1 & \Longleftrightarrow 0       \\
        + & \Longleftrightarrow \bullet
    \end{aligned}
    \right\} F'
\end{equation}

\subsection{常用公式}

\newpage

\section{逻辑函数标准形式}

\begin{enumerate}

    \item 最小项及标准与或式

          最小项:\textbf{与项}包含全部 n 个变量,全部以原变量或反变量的形式出现,且只出现一次

          例如:$A\overline{B}C$

          函数简写作:$\sum{m(0,1,2,3\dots)}$

    \item 最大项及标准或与式

          最大项:\textbf{或项}包含全部 n 个变量,全部以原变量或反变量的形式出现,且只出现一次

          例如:$A+\overline{B}+C$

          函数简写作:$\prod{M(0,1,2,3\dots)}$

    \item 两者转换

          \begin{enumerate}

              \item 代数转换法

                    \begin{enumerate}

                        \item 转换到最小项之和
                        \item 转换为一般与或式
                        \item 将非最小项拓展到最小项

                    \end{enumerate}

                    \begin{enumerate}

                        \item 转换到最大项之积
                        \item 转换为一般或与式
                        \item 将非最大项转换到最大项

                    \end{enumerate}

              \item 真值表转换法

                    \begin{enumerate}

                        \item 1:$\sum{m(0,2,4,6\dots)}$
                        \item 0:$\prod{M(1,3,5,7\dots)}$

                    \end{enumerate}

          \end{enumerate}

\end{enumerate}

\newpage

\section{逻辑函数化简}

\begin{enumerate}

    \item 代数化简化

          \begin{enumerate}

              \item 与或
              \item 并项法$AB+A\overline{B}=A$
              \item 吸收法$A+AB=A$
              \item 消去法$A+\overline{A}B=A+B$
              \item 配项法$A\cdot{1}=1,A+\overline{A}=1$
              \item 或与
              \item 定理法
              \item 求偶得到$F'(与或)$化简再求偶得到 F

          \end{enumerate}


    \item 卡诺图简化


          \begin{table}[!htbp]
              \centering
              \begin{tabular}{|r|c|c|c|c|}
                  \hline
                  \diagbox{AB}{CD} & 00                                                       & 01                                        & 11                             & 10                                          \\
                  \hline
                  00               & $\overline{A}\,\overline{B}\,\overline{C}\,\overline{D}$ & $\overline{A}B\overline{C}\,\overline{D}$ & $AB\overline{C}\,\overline{D}$ & $A\overline{B}\,\overline{C}\,\overline{D}$ \\
                  \hline
                  01               & $\overline{A}\,\overline{B}\,\overline{C}D$              & $\overline{A}B\overline{C}D$              & $AB\overline{C}D$              & $A\overline{B}\,\overline{C}D$              \\
                  \hline
                  11               & $\overline{A}\,\overline{B}CD$                           & $\overline{A}BCD$                         & $ABCD$                         & $A\overline{B}CD$                           \\
                  \hline
                  10               & $\overline{A}\,\overline{B}C\overline{D}$                & $\overline{A}BC\overline{D}$              & $ABC\overline{D}$              & $A\overline{B}C\overline{D}$                \\
                  \hline
              \end{tabular}
          \end{table}

          \begin{enumerate}
              \item $~AB+A\overline{B}=A$
              \item $\overline{A}B\overline{C}+\overline{A}BC+AB\overline{C}+ABC=B\overline{C}+BC=B$
          \end{enumerate}

          \begin{enumerate}
              \item $2^n$ 圈
              \item 卡诺圈尽量大
              \item 卡诺圈个数尽量少
              \item 每个 1 可以被多个卡诺圈包含
          \end{enumerate}

\end{enumerate}

\newpage
