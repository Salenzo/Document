\chapter{异步时序电路}
\newpage

\section{特性}
\begin{enumerate}
    \item 电路没有统一的时钟信号,电路状态改变直接由外部输入信号变化引起
    \item 类别:
        脉冲
        电平
    \item 每一时刻仅允许一个输入发生变化
    \item 只有电路进入一个新的稳定状态时才允许输入变化
    \item Mealy 模型:输出不仅与输入状态有关,还与二次状态有关
    \item Moore 模型:输出仅与二次状态有关
    \item 研究工具:
        脉冲:状态图,状态表
        电平:状态流程图,时序图
    \item 电平信号:基本信号
    \item 脉冲信号:连续两次电平跳变
\end{enumerate}


\section{脉冲异步分析与设计}

脉冲异步时序电路分析设计方法与同步时序电路相似,但有输入信号限制:
\begin{enumerate}
    \item 不允许同时出现两个及以上的输入脉冲
    \item 对于 $n$ 个输入端的电路,仅有 $n+1$ 种不同的输入信号组合
    \item 第二个脉冲到达必须在第一个脉冲引起的电路响应结束后 
\end{enumerate}

\section{电平异步分析与设计}

信号限制:
\begin{enumerate}
    \item 同时只允许一个输入电平发生变化,而且且一定要是相邻的
    \item 输入电平变化必须在第一个变化引起的电路响应结束后
\end{enumerate}

\begin{enumerate}
    \item 根据逻辑电路图,写出激励函数和输出函数表达式
    \item 列出状态流程表
    \item 作出时序图
    \item 说明电路逻辑功能
\end{enumerate}

\subsection{电平异步竞争与冒险}

\subsubsection{本质冒险}

输入的原变量与反变量由于延迟没有同时保持互补状态,输入信号通过反馈回路的延迟小于通过反相器的延迟,电路出现不正常转移。

要消除本质冒险,需要选择元件延迟特性或在反馈回路中加入足够的延迟。

\newpage
