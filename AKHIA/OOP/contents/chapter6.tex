\chapter{模板}
\newpage

\section{模板基础知识}

模板头:
\begin{lstlisting}[frame=shadowbox]
    \\ class == typename
    template<class T>
    template<typename T>
    template<typename T1,typename T2,typename T3>
\end{lstlisting}

模板类外定义函数:
\begin{lstlisting}[frame=shadowbox]
    template<class T>
    Array<T>::Array(int s){
        a = new T[size = s];
    }
\end{lstlisting}

模板实例:
\begin{lstlisting}[frame=shadowbox]
    Array<double> Array<int> etc
\end{lstlisting}

对象不属于模板类,只属于模板实例。

参数表中的模板类:模板类可以作为一种数据类型出现在参数表中。

模板的函数类型参数:
\begin{lstlisting}[frame=shadowbox]
    template<class T,int X,float Y>
\end{lstlisting}

\section{断言}

断言判断一个表达式,如果结果为假,输出诊断消息并中止程序。

\begin{lstlisting}[frame=shadowbox]
    void assert( 
       int expression 
    );
\end{lstlisting}

参数:Expression (including pointers) that evaluates to nonzero or 0.(表达式【包括指针】是非零或零)

原理:assert 的作用是现计算表达式 expression ,如果其值为假(即为0),那么它先向 stderr 打印一条出错信息,然后通过调用 abort 来终止程序运行。

用法总结:

\begin{enumerate}
    \item 在函数开始处检验传入参数的合法性
          如:
          \begin{lstlisting}[frame=shadowbox]
        int resetBufferSize(int nNewSize)
        {
          //功能:改变缓冲区大小,
          //参数:nNewSize 缓冲区新长度
        //返回值:缓冲区当前长度
        //说明:保持原信息内容不变     nNewSize<=0表示清除缓冲区
        assert(nNewSize >= 0);
        assert(nNewSize <= MAX_BUFFER_SIZE);
        }
        \end{lstlisting}
    \item 每个 assert 只检验一个条件,因为同时检验多个条件时,如果断言失败,无法直观的判断是哪个条件失败

    \item 不能使用改变环境的语句,因为assert只在DEBUG个生效

    \item assert 和后面的语句应空一行,以形成逻辑和视觉上的一致感

    \item 有的地方,assert 不能代替条件过滤
\end{enumerate}



ASSERT 只有在 Debug 版本中才有效,如果编译为 Release 版本则被忽略掉。
(在C中,ASSERT 是宏而不是函数),使用 ASSERT “断言”容易在 debug 时输出程序错误所在。
而 assert() 的功能类似,它是 ANSI C 标准中规定的函数,它与 ASSERT 的一个重要区别是可以用在 Release 版本中。

使用 assert 的缺点是,频繁的调用会极大的影响程序的性能,增加额外的开销。
在调试结束后,可以通过在包含 \#include <assert.h> 的语句之前插入 \#define NDEBUG 来禁用 assert 调用,
示例代码如下:
\begin{lstlisting}[frame=shadowbox]
        #include <stdio.h>
        #define NDEBUG
        #include <assert.h>
    \end{lstlisting}


\section{STL 标准模板库}

\newpage