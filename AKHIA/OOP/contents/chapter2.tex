\chapter{类}
\newpage

\section{构造函数}

\subsection{基本特征}

构造函数是与类名相同的成员函数。

编译器默认添加:
\begin{enumerate}
    \item 构造函数
    \item 拷贝构造函数
    \item 析构函数
    \item 赋值运算符函数
\end{enumerate}

\begin{enumerate}
    \item 返回类型为void
    \item 可以重载,但必须有不同的函数署名
    \item 默认不带任何参数
    \item 创建对象时会隐式调用
    \item 用来初始化数据成员
    \item 默认构造函数定义在类内
    \item 带参构造函数定义在类外
\end{enumerate}

\subsection{拷贝构造函数}

如果不提供,编译器会自动生成:将源对象所有数据成员的值逐一赋值给目标对象相应的数据成员。

\begin{lstlisting}[frame=shadowbox]
    Person(Person&);
    Person(const Person&);
\end{lstlisting}

\subsection{转型构造函数}

关闭因转型构造函数导致的隐式类型转换,将运行期错误变成了编译器错误。

\begin{lstlisting}[frame=shadowbox]
    explicit Person(const string& n) {name = n;}    
\end{lstlisting}

\subsection{构造函数初始化}

\begin{lstlisting}[frame=shadowbox]
    class C {
        public:
            C() {
                x = 0;
                c = 0; //ERROR(CONST)
            }
        private:
            int x;
            const int c;
    }    
\end{lstlisting}

对const类型初始化,只需要添加一个\textbf{初始化列表}:

\begin{lstlisting}[frame=shadowbox]
    class C {
        public:
            C() : c(0) {x = 0;}
        private:
            int x;
            const int c;
    }    
\end{lstlisting}

\textbf{初始化段}由冒号:开始,c为需要初始化的数据成员,()内是初始值,这是初始化const的\textbf{唯一方法}。
\textbf{初始化列表}仅在\textbf{构造函数}中有效。
数据成员的顺序仅取决于类中的顺序,与初始化段中的顺序无关。


\newpage

\subsection{new}

\begin{lstlisting}[frame=shadowbox]

    new constructor[([arguments])]

\end{lstlisting}

\section{析构函数}

析构函数当对象被\textbf{销毁}时,自动调用。

\begin{lstlisting}[frame=shadowbox]
    class C {
        public:
            C() {}
            ~C() {}
        private:
            int x;
\end{lstlisting}

没有\textbf{参数}和\textbf{返回值},不能重载。

\section{指向对象的指针}

\begin{table}[!htbp]
    \centering
    \begin{tabular}{lccc}
        \toprule
        名称           & 符号 & 用途                         \\
        \midrule
        成员选择操作符 & .    & 对象和对象引用               \\
        指针操作符     & $->$ & 指针访问成员,专用于对象指针 \\
        \bottomrule
    \end{tabular}
\end{table}

用途:
\begin{enumerate}
    \item 作为参数传递给函数,通过函数返回
    \item 使用new([])操作符动态创建对象,然后返回对象的指针
\end{enumerate}

\subsection{常量指针 this}

常量指针:不能赋值、递增、递减,不能在static成员函数中使用。

避免命名冲突:
\begin{lstlisting}[frame=shadowbox]
    void setID (const string& id) {this->id=id;}
\end{lstlisting}

\section{类数据成员和类成员函数}

如果不使用 static(静态) 关键字,数据成员和成员函数都是属于\textbf{对象}的。
而使用 static 则可以创建\textbf{类成员}:分为对象成员和实例成员。

\subsection{类数据成员}

静态成员只与类本身有关,与对象无关。它对整个类而言只有一个,而且必须在\textbf{任何程序块外}定义。

\subsection{类成员函数}

static 成员函数只能访问其他 static 成员,而非 static 成员都可以访问。
同时 static 成员函数也可以是 inline 函数。

可以通过两种方式访问:
\begin{lstlisting}[frame=shadowbox]
    C c1;
    c1.Meth();
    C::Meth();
    unsigned x = c1.var;
    unsigned y =C::var;
\end{lstlisting}

但首选用类直接访问,为了说明静态成员直接与类关联。

\subsection{成员函数中的静态变量}

如果将成员函数中的局部变量定义为静态变量,\textbf{类的所有对象在调用这个成员函数共享这个变量}。


\section{友元(friend)类}

友元类中的所有成员函数都是目标类中的友元函数。

\newpage