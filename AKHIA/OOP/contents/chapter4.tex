\chapter{多态}
\newpage

\section{运行期绑定和编译器绑定}

多态的函数的绑定是\textbf{运行期绑定}而不是\textbf{编译期绑定}。

\subsection{多态的前提}

\begin{enumerate}
    \item 必须存在一个继承体系结构
    \item 继承体系结构中的一些类必须拥有同名的 virtual 函数
    \item 至少有一个基类类型的指针或基类类型的引用,这个指针或引用可用来对 virtual 成员函数进行调用
\end{enumerate}

基类类型的指针可以指向任何基类对象和派生类对象。
虚函数在系统中的运行期才对调用动作绑定。

\subsection{虚函数成员继承}

为了调用同一基类的类族的同名函数,不用定义多个指向各派生类的指针变量,而是使用虚函数用同一方式调用同一类族中不同类的所有同名函数。
和普通成员函数相同,在派生类中虚成员函数也可以从基类中继承。

\subsection{运行期绑定与虚成员函数表}

vtable 虚成员函数表用来实现虚成员函数的运行期绑定。用途是支持运行期查询,使系统可以将一函数名绑定到虚成员函数表的特定入口地址。
而如果在基类中重新定义,则会调用基类中的成员函数,无法达到多态效果。虚成员函数表需要额外的空间,查询也需要时间。
构造函数不能是虚函数,析构函数可以是虚函数。

\newpage

\section{重载、覆盖和隐藏}

只有虚函数在运行期绑定,是真正的多态。所以编译期多态机制与运行期多态机制相似,但也有区别。

\begin{table}[!htbp]
    \centering
    \begin{tabular}{lcccc}
        \toprule
                            & 有效域                                & 函数     & 同              & 异       \\
        \midrule
        重载                & 层次类                                & 虚 or 实 & 函数名          & 函数签名 \\
        \hline
        覆盖                & 层次类                                & 虚       & 函数名 and 签名 &          \\
        \hline
        \multirow{2}*{隐藏} & \multirow{2}*{顶层 or 同类内成员函数} & 实       & 函数名          & 函数签名 \\
                            &                                       & 虚       & 函数名 and 签名 &          \\
        \bottomrule
    \end{tabular}
\end{table}

\begin{enumerate}
    \item 重载
          同类中的成员函数或顶层函数签名不同而同名则称重载,属于编译器绑定。
    \item 覆盖
          派生类中虚成员函数与基类中函数成员同名且签名相同则称覆盖。
    \item 隐藏(要避免)
          同名而函数签名不同的层次类。
\end{enumerate}

\subsection{名字共享}

\begin{enumerate}
    \item 重载函数名的顶层函数
    \item 重载构造函数
    \item 同一类中非构造函数名字相同
    \item 继承层次中的同名函数(特别是虚函数)
\end{enumerate}

\section{抽象基类}

抽象基类不能实例化抽象基类的对象,用来指明派生类必须覆盖某些虚函数才能拥有对象。

抽象基类:必须拥有至少一个纯虚函数。

\subsection{纯虚函数}

在虚成员函数声明的结尾加上$=0$即可将其定义为纯虚函数。

\begin{lstlisting}[frame=shadowbox]
    virtual void open() = 0;
\end{lstlisting}

\subsection{公共接口}

抽象基类定义了一个公共接口,被所有从抽象基类派生的类共享。
由于抽象基类只含有 public 成员,通常使用关键字 struct 声明抽象基类。

\section{运行期类型识别(RTTI)}

\begin{enumerate}
    \item 在运行期对类型转换进行检查
    \item 在运行期确定对象类型
    \item 拓展 RTTI 
\end{enumerate}

\subsection{dynamic\_cast 操作符}

在运行期对可能引发错误的转型操作进行测试。

static\_cast 不做运行时检查,可能不安全。
\begin{lstlisting}[frame=shadowbox]
    class B {}
    class D : public B {}
    D* p;
    p = new B;
    p = static_cast<D*>(new B);
\end{lstlisting}

而 dynamic\_cast 虽然语法相同,但仅对多态类型(至少有一个虚函数)有效。
dynamic\_cast 可以实现向上、向下转型。

static\_cast 可以施加于任何类型,无论是否有多态性。
dynamic\_cast 只能施加于具有多态性的类型,转型的目的类型只能是指针或引用。
dynamic\_cast 虽然运行范围没有 static\_cast 广,但是只有 dynamic\_cast 能进行运行期安全检查。

\subsection{typeid 操作符}

确定\textbf{表达式}的类型。

\begin{lstlisting}[frame=shadowbox]
#inlcude <typeinfo>
\end{lstlisting}

\newpage