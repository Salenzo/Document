\chapter{単語}
\newpage

\begin{enumerate}
    \item 単語
    \begin{enumerate}
        \item 自立語
        \begin{enumerate}
            \item 活用がある
            \begin{enumerate}
                \item 述語(用言)
                \begin{enumerate}
                    \item ウ段の音で終わる ➞ 動詞
                    \item 「い」で終わる ➞ 形容詞
                    \item 「だ」で終わる ➞ 形容動詞
                \end{enumerate}}
            \end{enumerate}
            \item 活用がない
            \begin{enumerate}
                \item 主語 ➞ 名詞(代名詞)
                \item 修飾語
                \begin{enumerate}
                    \item 用言 ➞ 副詞
                    \item 体言 ➞ 連体詞
                \end{enumerate}
                \item 接続語 ➞ 接続詞
                \item 独立語 ➞ 感動詞
            \end{enumerate}
        \end{enumerate}
        \item 付属語
        \begin{enumerate}
            \item 活用がある ➞ 助動詞
            \item 活用がない ➞ 助詞
        \end{enumerate}
    \end{enumerate}
\end{enumerate}

\section{自立語}

有明确的词汇意义。

\section{付属語}

纯粹的语法功能,需要依附于前者才有意义。

\section{活用}

所谓活用,就是变形。
能够活用(变形)的词不多,有四种:\textbf{动词、形容词、形容动词、助动词}。
前三者是自立语,而助动词不是。

\section{用言}

动词、形容词、形容动词统称用言。
何为用言?就是能够活用的自立语。

此外,通过图不难发现,它能当述語。
所谓述語,就是放在句子最后的\textbf{动词、形容词、形容动词、以及名词+断定助动词}。

「い」结尾的是形容词;「だ」结尾的是形容动词;「ウ段」结尾的是动词。

\section{词干·词尾}

活用中保持形态不变的部分叫做词干,而变化的部分叫做词尾。

\section{形容词}

形容词就是词尾是「い」的用言。标日认为这货叫【形容词1】,也被叫做【い形形容词】

参照上面的活用表,我们来理解一下各个活用:

未然:美しかろう,丁宁体则是【美しいでしょう】,表推量。

连用1:美しく(て),表中顿,标日称【て形】。也可修饰动词,如【美しくなる】;
    标日则会这么说,这是形容词1【美しい】的【て形】去掉【て】后再加上动词【なる】。
    而学校语法则说,这是形容词【美しい】的连用形加动词【なる】。
    这么一看,学校语法的优越性就体现出来了对吧!

连用2:美しかっ(た),表过去,标日称【た形】

终止:放在句末的【美しい】,作用是作【述語】。简单的说,终止就是结句。形容词是有能力结句的。

连体:美しい(こと),作用是连接名词、修饰名词。不难发现【终止】和【连体】在动词和形容词中是高度一致的。
    因此标日统称为普通型、原型。

命令:形容词和形容动词没有命令形。


形容动词

形容词就是词尾是「だ」的用言。标日认为这货叫【形容词2】,也被叫做【な形形容词】,因为从意义上看都是形容词。

虽说词尾是「だ」,但是在大多数时候字典中是仅收录【词干】的。

这货的定位是比较尴尬的,初期在活用上有不少动词的特点,到了近代则是名词化倾向严重。

据说只有学校语法认可【形容动词】这一名称。

这次我们主要来关注连用、终止、连体三个活用:

连用1:静かで、表中顿。标日称之为【で形】

连用2:静かに(なる)、连接动词,修饰动词。

连用3:静かだった、过去。

终止:静かだ,结句。

连体:静かな(ところ)、连接名词,修饰名词。标日称【な形】



日语中,用言和体言的区别,最根本的区别就是两者内容/用法不同,
日语中,用言(有3种即“动词”、“形容词”、“形容动词”的总称,叫做“用言”),
而体言(日语中的名词和代词)。

日语的品词共有12种,其中有3种即“动词”、“形容词”、“形容动词”的总称,叫做“用言”。
所谓用言,就是有“活用”的独立品词。它用来表示事物的动作、存在、性质、状态等属性。
虽然助动词也有活用,但是它只是附属词,所以不列入用言之列。

\section{v}

\newpage

\subsection{i}

\newpage

\subsection{ii}

\newpage

\subsection{iii}

\newpage

\section{n}

\newpage

\subsection{pron}

\newpage

\section{adj}

\newpage

\section{adv}

\newpage

\section{conj}

\newpage

\section{接続}

\newpage