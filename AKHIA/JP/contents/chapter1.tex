\chapter{単語}
\newpage

\begin{enumerate}
    \item 単語
          \begin{enumerate}
              \item 自立語
                    \begin{enumerate}
                        \item 活用がある\par
                              述語(用言)
                              \begin{enumerate}
                                  \item ウ段の音で終わる ➞ 動詞
                                  \item 「い」で終わる ➞ 形容詞
                                  \item 「だ」で終わる ➞ 形容動詞
                              \end{enumerate}
                        \item 活用がない
                              \begin{enumerate}
                                  \item 主語 ➞ 名詞(代名詞)
                                  \item 修飾語\par
                                        用言 ➞ 副詞\par
                                        体言 ➞ 連体詞\par
                                  \item 接続語 ➞ 接続詞
                                  \item 独立語 ➞ 感動詞
                              \end{enumerate}
                    \end{enumerate}
              \item 付属語
                    \begin{enumerate}
                        \item 活用がある ➞ 助動詞
                        \item 活用がない ➞ 助詞
                    \end{enumerate}
          \end{enumerate}
\end{enumerate}

\section{自立語}

有明确的词汇意义。

\section{付属語}

纯粹的语法功能,需要依附于前者才有意义。

\section{活用}

所谓活用,就是变形。
能够活用(变形)的词不多,有四种:\textbf{动词、形容词、形容动词、助动词}。
前三者是自立语,而助动词不是。

\section{用言}

动词、形容词、形容动词统称用言。
何为用言?就是能够活用的自立语。

此外,通过图不难发现,它能当述語。
所谓述語,就是放在句子最后的\textbf{动词、形容词、形容动词、以及名词+断定助动词}。

「い」结尾的是形容词;「だ」结尾的是形容动词;「ウ段」结尾的是动词。

\section{词干·词尾}

活用中保持形态不变的部分叫做词干,而变化的部分叫做词尾。

\section{形容词}

形容词就是词尾是「い」的用言,也被叫做【い形形容词】。

\begin{enumerate}
    \item 未然:美しかろう,丁宁体则是【美しいでしょう】,表推量。
    \item 连用1:美しく(て),表中顿,标日称【て形】。也可修饰动词,如【美しくなる】;
          标日则会这么说,这是形容词1【美しい】的【て形】去掉【て】后再加上动词【なる】。
          而学校语法则说,这是形容词【美しい】的连用形加动词【なる】。
          这么一看,学校语法的优越性就体现出来了对吧!
    \item 连用2:美しかっ(た),表过去,标日称【た形】
    \item 终止:放在句末的【美しい】,作用是作【述語】。简单的说,终止就是结句。形容词是有能力结句的。
    \item 连体:美しい(こと),作用是连接名词、修饰名词。不难发现【终止】和【连体】在动词和形容词中是高度一致的。
          因此标日统称为普通型、原型。
    \item 命令:形容词和形容动词没有命令形。
\end{enumerate}

\section{形容动词}

形容词就是词尾是「だ」的用言。标日认为这货叫【形容词2】,也被叫做【な形形容词】,因为从意义上看都是形容词。
虽说词尾是「だ」,但是在大多数时候字典中是仅收录【词干】的。它的定位是比较尴尬的,初期在活用上有不少动词的特点,
到了近代则是名词化倾向严重。


主要关注连用、终止、连体三个活用:
\begin{enumerate}
    \item 连用1:静かで、表中顿。标日称之为【で形】
    \item 连用2:静かに(なる)、连接动词,修饰动词。
    \item 连用3:静かだった、过去。
    \item 终止: 静かだ,结句。
    \item 连体: 静かな(ところ)、连接名词,修饰名词。标日称【な形】
\end{enumerate}

日语中,用言和体言的区别,最根本的区别就是两者内容/用法不同,用言(有3种即“动词”、“形容词”、“形容动词”的总称,叫做“用言”),
而体言(日语中的名词和代词)。日语的品词共有12种,其中有3种即“动词”、“形容词”、“形容动词”的总称,叫做“用言”。所谓用言,
就是有“活用”的独立品词。它用来表示事物的动作、存在、性质、状态等属性。虽然助动词也有活用,但是它只是附属词,所以不列入用言之列。

\section{動詞}

\newpage

\subsection{五段動詞}

\newpage

\subsection{カ変動詞}

\newpage

\subsection{サ変動詞}

\newpage

\subsection{形容動詞}

\newpage

\section{形容詞}

\newpage

\section{名詞}

\newpage

\subsection{代名詞}

\newpage

\section{副詞}

\newpage

\section{連体詞}

\newpage

\section{接続词}

\newpage

\section{感動詞}

\newpage

\section{助動詞}

\newpage

\section{助詞}




\newpage