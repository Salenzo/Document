\chapter{文法}
\newpage

\section{文型}

  \subsection{判断句}

    以体言(名词、代词)作谓语的句子。
    
    例:私は田中です。
    —— 「田中」作谓语

    ~は~です。
    ~は~じゃ(では)ありません。
    ~は~ですか。
    ~も
    ~の~
    ~さん
    その・それ・あれ
    その~・その~・あの~
    そうです。
    ~か、~か
    ~の~
    「の」代替名词使用
    お~
    そうですか
    ここ・そこ・あそこ|こちら・そちら・あちら
    ~は~「place」です
    どこ・どちら
    ~の~
    こ・そ・あ・ど
    お~
    いま ~時~分です
    ~ます・~ません・~ました・~ませんでした
    ~に~
    ~から~まで
    ~と~
    ~ね
    ~へ いきます・来ます・帰ります
    どこ「へ」も いきません・行きませんでした
    ~で いきます・来ます・帰ります
    ~と ~
    いつ
    ~よ
    そうですね。
    ~を~
    ~を します。
    何を しますか。
    なん・なに
    ~で~
    ~で~
    ~ませんか。
    ~ましょう。
    ~か

  \subsection{叙述句}

    以动词作谓语的句子。
    
    例:李さんは飲み物を買います。
    —— 「買う」作谓语


  \subsection{描写句}

    以形容词、形容动词作谓语的句子。
    
    例:富士山は美しい。
    —— 「美しい」作谓语

    \subsection{存在句}
    
    表示存在关系的句子,经常以「ある/いる」作谓语。
    
    例:王さんは今教室にいます。

    \section{三维度}

    日语句子按照时态划分,分为:现代式、过去式;
    按照语体划分,分为:敬体形、简体形;
    按照表达的意思划分,分为:肯定、否定。
    由于三个维度的存在,使得日语句子会呈现不同的变形。

    \section{时态}

    \section{语体}

    \section{表意}

\newpage