\chapter{Dockerfile}
\newpage

\section{什么是 Dockerfile?}

Dockerfile 是一个用来构建镜像的文本文件,文本内容包含了一条条构建镜像所需的指令和说明。
使用 Dockerfile 定制镜像

这里仅讲解如何运行 Dockerfile 文件来定制一个镜像,具体 Dockerfile 文件内指令详解,将在下一节中介绍,这里你只要知道构建的流程即可。

1、下面以定制一个 nginx 镜像(构建好的镜像内会有一个 /usr/share/nginx/html/index.html 文件)

在一个空目录下,新建一个名为 Dockerfile 文件,并在文件内添加以下内容:

FROM nginx
RUN echo '这是一个本地构建的nginx镜像' > /usr/share/nginx/html/index.html

2、FROM 和 RUN 指令的作用

FROM:定制的镜像都是基于 FROM 的镜像,这里的 nginx 就是定制需要的基础镜像。后续的操作都是基于 nginx。

RUN:用于执行后面跟着的命令行命令。有以下俩种格式:

shell 格式:

RUN <命令行命令>
# <命令行命令> 等同于,在终端操作的 shell 命令。

exec 格式:

RUN ["可执行文件", "参数1", "参数2"]
# 例如:
# RUN ["./test.php", "dev", "offline"] 等价于 RUN ./test.php dev offline

注意:Dockerfile 的指令每执行一次都会在 docker 上新建一层。所以过多无意义的层,会造成镜像膨胀过大。例如:
FROM centos
RUN yum install wget
RUN wget -O redis.tar.gz "http://download.redis.io/releases/redis-5.0.3.tar.gz"
RUN tar -xvf redis.tar.gz
以上执行会创建 3 层镜像。可简化为以下格式:
FROM centos
RUN yum install wget \
    && wget -O redis.tar.gz "http://download.redis.io/releases/redis-5.0.3.tar.gz" \
    && tar -xvf redis.tar.gz

如上,以 && 符号连接命令,这样执行后,只会创建 1 层镜像。
开始构建镜像

在 Dockerfile 文件的存放目录下,执行构建动作。

以下示例,通过目录下的 Dockerfile 构建一个 nginx:v3(镜像名称:镜像标签)。

注:最后的 . 代表本次执行的上下文路径,下一节会介绍。
$ docker build -t nginx:v3 .

以上显示,说明已经构建成功。
上下文路径

上一节中,有提到指令最后一个 . 是上下文路径,那么什么是上下文路径呢?
$ docker build -t nginx:v3 .

上下文路径,是指 docker 在构建镜像,有时候想要使用到本机的文件(比如复制),docker build 命令得知这个路径后,会将路径下的所有内容打包。

解析:由于 docker 的运行模式是 C/S。我们本机是 C,docker 引擎是 S。实际的构建过程是在 docker 引擎下完成的,所以这个时候无法用到我们本机的文件。这就需要把我们本机的指定目录下的文件一起打包提供给 docker 引擎使用。

如果未说明最后一个参数,那么默认上下文路径就是 Dockerfile 所在的位置。

注意:上下文路径下不要放无用的文件,因为会一起打包发送给 docker 引擎,如果文件过多会造成过程缓慢。
指令详解
COPY

复制指令,从上下文目录中复制文件或者目录到容器里指定路径。

格式:

COPY [--chown=<user>:<group>] <源路径1>...  <目标路径>
COPY [--chown=<user>:<group>] ["<源路径1>",...  "<目标路径>"]

[--chown=<user>:<group>]:可选参数,用户改变复制到容器内文件的拥有者和属组。

<源路径>:源文件或者源目录,这里可以是通配符表达式,其通配符规则要满足 Go 的 filepath.Match 规则。例如:

COPY hom* /mydir/
COPY hom?.txt /mydir/

<目标路径>:容器内的指定路径,该路径不用事先建好,路径不存在的话,会自动创建。
ADD

ADD 指令和 COPY 的使用格式一致(同样需求下,官方推荐使用 COPY)。功能也类似,不同之处如下:

    ADD 的优点:在执行 <源文件> 为 tar 压缩文件的话,压缩格式为 gzip, bzip2 以及 xz 的情况下,会自动复制并解压到 <目标路径>。
    ADD 的缺点:在不解压的前提下,无法复制 tar 压缩文件。会令镜像构建缓存失效,从而可能会令镜像构建变得比较缓慢。具体是否使用,可以根据是否需要自动解压来决定。

CMD

类似于 RUN 指令,用于运行程序,但二者运行的时间点不同:

    CMD 在docker run 时运行。
    RUN 是在 docker build。

作用:为启动的容器指定默认要运行的程序,程序运行结束,容器也就结束。CMD 指令指定的程序可被 docker run 命令行参数中指定要运行的程序所覆盖。

注意:如果 Dockerfile 中如果存在多个 CMD 指令,仅最后一个生效。

格式:

CMD <shell 命令> 
CMD ["<可执行文件或命令>","<param1>","<param2>",...] 
CMD ["<param1>","<param2>",...]  # 该写法是为 ENTRYPOINT 指令指定的程序提供默认参数

推荐使用第二种格式,执行过程比较明确。第一种格式实际上在运行的过程中也会自动转换成第二种格式运行,并且默认可执行文件是 sh。
ENTRYPOINT

类似于 CMD 指令,但其不会被 docker run 的命令行参数指定的指令所覆盖,而且这些命令行参数会被当作参数送给 ENTRYPOINT 指令指定的程序。

但是, 如果运行 docker run 时使用了 --entrypoint 选项,此选项的参数可当作要运行的程序覆盖 ENTRYPOINT 指令指定的程序。

优点:在执行 docker run 的时候可以指定 ENTRYPOINT 运行所需的参数。

注意:如果 Dockerfile 中如果存在多个 ENTRYPOINT 指令,仅最后一个生效。

格式:

ENTRYPOINT ["<executeable>","<param1>","<param2>",...]

可以搭配 CMD 命令使用:一般是变参才会使用 CMD ,这里的 CMD 等于是在给 ENTRYPOINT 传参,以下示例会提到。

示例:

假设已通过 Dockerfile 构建了 nginx:test 镜像:

FROM nginx

ENTRYPOINT ["nginx", "-c"] # 定参
CMD ["/etc/nginx/nginx.conf"] # 变参 

1、不传参运行

$ docker run  nginx:test

容器内会默认运行以下命令,启动主进程。

nginx -c /etc/nginx/nginx.conf

2、传参运行

$ docker run  nginx:test -c /etc/nginx/new.conf

容器内会默认运行以下命令,启动主进程(/etc/nginx/new.conf:假设容器内已有此文件)

nginx -c /etc/nginx/new.conf

ENV

设置环境变量,定义了环境变量,那么在后续的指令中,就可以使用这个环境变量。

格式:

ENV <key> <value>
ENV <key1>=<value1> <key2>=<value2>...

以下示例设置 NODE_VERSION = 7.2.0 , 在后续的指令中可以通过 $NODE_VERSION 引用:

ENV NODE_VERSION 7.2.0

RUN curl -SLO "https://nodejs.org/dist/v$NODE_VERSION/node-v$NODE_VERSION-linux-x64.tar.xz" \
  && curl -SLO "https://nodejs.org/dist/v$NODE_VERSION/SHASUMS256.txt.asc"

ARG

构建参数,与 ENV 作用一至。不过作用域不一样。ARG 设置的环境变量仅对 Dockerfile 内有效,也就是说只有 docker build 的过程中有效,构建好的镜像内不存在此环境变量。

构建命令 docker build 中可以用 --build-arg <参数名>=<值> 来覆盖。

格式:

ARG <参数名>[=<默认值>]

VOLUME

定义匿名数据卷。在启动容器时忘记挂载数据卷,会自动挂载到匿名卷。

作用:

    避免重要的数据,因容器重启而丢失,这是非常致命的。
    避免容器不断变大。

格式:

VOLUME ["<路径1>", "<路径2>"...]
VOLUME <路径>

在启动容器 docker run 的时候,我们可以通过 -v 参数修改挂载点。
EXPOSE

仅仅只是声明端口。

作用:

    帮助镜像使用者理解这个镜像服务的守护端口,以方便配置映射。
    在运行时使用随机端口映射时,也就是 docker run -P 时,会自动随机映射 EXPOSE 的端口。

格式:

EXPOSE <端口1> [<端口2>...]

WORKDIR

指定工作目录。用 WORKDIR 指定的工作目录,会在构建镜像的每一层中都存在。(WORKDIR 指定的工作目录,必须是提前创建好的)。

docker build 构建镜像过程中的,每一个 RUN 命令都是新建的一层。只有通过 WORKDIR 创建的目录才会一直存在。

格式:

WORKDIR <工作目录路径>

USER

用于指定执行后续命令的用户和用户组,这边只是切换后续命令执行的用户(用户和用户组必须提前已经存在)。

格式:

USER <用户名>[:<用户组>]

HEALTHCHECK

用于指定某个程序或者指令来监控 docker 容器服务的运行状态。

格式:

HEALTHCHECK [选项] CMD <命令>:设置检查容器健康状况的命令
HEALTHCHECK NONE:如果基础镜像有健康检查指令,使用这行可以屏蔽掉其健康检查指令

HEALTHCHECK [选项] CMD <命令> : 这边 CMD 后面跟随的命令使用,可以参考 CMD 的用法。

ONBUILD

用于延迟构建命令的执行。简单的说,就是 Dockerfile 里用 ONBUILD 指定的命令,在本次构建镜像的过程中不会执行(假设镜像为 test-build)。当有新的 Dockerfile 使用了之前构建的镜像 FROM test-build ,这是执行新镜像的 Dockerfile 构建时候,会执行 test-build 的 Dockerfile 里的 ONBUILD 指定的命令。

格式:

ONBUILD <其它指令>

\newpage