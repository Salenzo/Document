\chapter{基础}
\newpage

\section{二进制 八进制 十进制 十六进制 相互转化}

\subsection{十进制和其他进制的相互转换}

\begin{enumerate}

    \item 其他进制转换为十进制:

          各进制数按权展开并相加

    \item 十进制转换为其他进制:

          \begin{enumerate}

              \item 整数:除以基数取余,直到商为零,逆序
              \item 小数:乘以基数取整, 顺序

          \end{enumerate}

\end{enumerate}

\subsection{二进制 八进制 十六进制 的相互转换}

以小数点为界向两侧划分,按基数划分组,不够则补零。

\subsection{8421BCD 码 格雷码 余 3 码 与十进制之间的转换}

\begin{enumerate}

    \item 8421BCD
    \item G
    \item 余 3

\end{enumerate}

\section{十进制与 原码 反码 补码 之间的转换}

符号位 0,正数反码补码和原码相同。

符号位 1,负数反码数值取反,补码在反码最低有效位上加一。

\section{校验法}

\begin{enumerate}

    \item 奇偶校验码:可以验证传输过程是否产生了错误
    \item 奇校验:为二进制添加一位校验码,使 1 的数量为奇数
    \item 偶校验:为二进制添加一位校验码,使 1 的数量为偶数
    \item 海明码:传输过程中错一位概率大,通过海明码可以验证是哪位出错

\end{enumerate}

\newpage
