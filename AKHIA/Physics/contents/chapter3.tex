\chapter{波动光学}

\newpage

\section{光的本质}

光波是电磁波

同一媒质中的相对光强:$I={E_0}^2$

\section{光的相干性}

\subsection{发光机制}

\subsubsection{光源}

\begin{enumerate}
    \item 普通光源
          \begin{enumerate}
              \item 热光源:热能激发原子能级跃迁
              \item 冷光源:化学能,电能等激发
          \end{enumerate}
    \item 激光光源
\end{enumerate}

原子发光特点:
\begin{enumerate}
    \item 随机性
    \item 间歇性
    \item 各原子各级发光独立,波列互不相干
    \item 不相干性(独立光源不可能是一对相干光源:原子发光间歇而随机,振动方向和相位差不可能相同)
\end{enumerate}

\subsection{相干光源}

相干光源条件:
\begin{enumerate}
    \item 振动\textbf{频率}相同
    \item 振动\textbf{方向}相同
    \item \textbf{相位差}恒定
\end{enumerate}

原子自发辐射的间断性和相位随机性,不利于实现干涉条件。

\begin{equation}
    x_1+x_2=\sqrt{A_1^2+A_2^2+2A_1A_2\cos{(\varphi{}_2-\varphi{}_1})}\cos{(\omega{}t+\varphi{})}
\end{equation}

相长、相消:
\begin{equation}
    \begin{aligned}
        \delta{} & =r_2-r_1=\pm{}k\lambda      \\
        \delta{} & =r_2-r_1=\pm{}(2k+1)\lambda
    \end{aligned}
\end{equation}

\subsection{波动几何描述}

\begin{enumerate}
    \item 波线
    \item 波面
    \item 平面波
    \item 球面波
\end{enumerate}

\section{惠更斯原理}

惠更斯原理:媒质中波动到的各点,都可以看作新波源,子波的包络面就是该时刻的波面。

\subsection{相干光的获得}

干涉光的获得:
\begin{enumerate}
    \item 分波面法
    \item 分振幅法
\end{enumerate}

\section{杨氏双缝实验}

\subsection{明暗条纹位置的推导}

\subsubsection{明纹条件}

\begin{equation}
    \begin{aligned}
        \delta{}  & =r_2-r_1=d\sin{\theta}\approx{d\tan{\theta}}                          \\
                  & =\frac{xd}{D}=k\lambda{}                                              \\
        x         & =k\frac{D\lambda}{d}                         & k=0,\pm{1},\pm{2}\dots \\
        \Delta{x} & =\frac{D\lambda}{d}
    \end{aligned}
\end{equation}

\subsubsection{暗纹条件}

\begin{equation}
    \begin{aligned}
        \mu{}\pm\delta{} & =\frac{xd}{D}=(2k+1)\frac{\lambda{}}{2}                \\
        x                & =(2k+1)\frac{D\lambda}{2d}              & k=0,1,2\dots \\
        \Delta{x}        & =\frac{D\lambda}{d}
    \end{aligned}
\end{equation}

\subsection{光程}


真空光速:$\mathrm{C}$

光在介质中的速度:$v=\frac{\mathrm{C}}{n}$

真空中:$\mathrm{\lambda{}_0}=\frac{\mathrm{C}}{\nu}$

\begin{equation}
    \lambda{}=\frac{v}{\nu}=\frac{\mathrm{C}}{n\nu}=\frac{\mathrm{\lambda_0}}{n}
\end{equation}

$\lambda=\frac{\mathrm{\lambda_0}}{n}$

$\frac{x}{\lambda}=\frac{x}{\frac{\mathrm{\lambda_0}}{n}}=\frac{xn}{\mathrm{\lambda_0}}$

$(r_2-t)+nt=r_2+(n-1)$

$r_2=(r_1-d)+nd$

\section{薄膜干涉}

当光从折射率小的光疏介质,正入射或掠入射于折射率大的光密介质时,则反射光有半波损失。
当光从折射率大的光密介质,正入射于折射率小的光疏介质时,反射光没有半波损失。
折射光没有相位突变。

若厚度 $e$ 一定,则相同入射角入射的光束,经膜的上下表面反射后产生的相干光束都有相同的光程差,
从而对应于干涉图样中的一条条纹,此称干涉为\textbf{等倾干涉}(不要求)
若入射角 i 一定,则称此干涉为\textbf{等厚干涉}(重点)
薄膜厚度在 $10^{-7}m$ 数量级。

相位相差了 $\pi$ 相当于波程差了 $\frac{\lambda}{2}$,称为半波损失。

\begin{equation}
    \delta=\delta_0+\delta{}'=2\mathrm{e}\sqrt{n_2^2-n_1^2\sin^2{i}}+
    \left\{
    \begin{aligned}
         & \frac{\lambda}{2} & \mbox{反射条件不同} \\
         & 0                 & \mbox{反射条件相同}
    \end{aligned}
    \right.
\end{equation}

明文暗纹:
\begin{equation}
    \delta_{r}=
    \left\{
    \begin{aligned}
        \mathrm{k}\lambda{}             ~~ & (k=1,2,3\dots) \\
        \frac{2\mathrm{k}+1}{2}\lambda{}~~ & (k=0,1,2\dots)
    \end{aligned}
    \right.
\end{equation}

$k$ 的取值要注意:如 $k=3$ 则表示明纹 3 条,暗纹 4 条。



\subsection{薄膜干涉的应用}

增透膜:现代光学仪器中,为了减少入射光能量,在仪器表面上反射时所引起的损失,常在仪器表面上镀一层厚度均匀的薄膜。

\section{等厚干涉}

\subsection{劈尖干涉}

$\theta{}\approx{}10^{-4}\sim{}10^{-5} \mathrm{rad}$

\begin{equation}
    \delta=2e+\frac{\lambda}{2}=
    \left\{
    \begin{aligned}
         & k\lambda{}              & , & k=1,2,3,\dots \\
         & \frac{2k+1}{2}\lambda{} & , & k=0,1,2,\dots
    \end{aligned}
    \right.
\end{equation}

$e=0$ 时是暗纹,证明了半波损失的存在。

\subsubsection{条纹与厚度关系}

相邻两条明纹暗纹间的厚度差:
\begin{equation}
    \begin{aligned}
        2\mathrm{e_k}     & +\frac{\lambda}{2}=  k\lambda     \\
        2\mathrm{e_{k+1}} & +\frac{\lambda}{2}=  (k+1)\lambda
    \end{aligned}\\
    \mbox{条纹间距:} \mathrm{e_{k+1}}-\mathrm{e_{k}}=\frac{\lambda}{2}
\end{equation}

\subsubsection{干涉条纹移动}

\begin{enumerate}
    \item $e$ 变大,条纹下移
    \item $e$ 变小,条纹上移
\end{enumerate}

\subsubsection{劈尖干涉的应用}

测膜厚:$e=k\frac{\lambda}{2n}~~~k=\underline{0},1,2,3,\dots$

检验光学元件平整度:
\begin{enumerate}
    \item 外弯:工件凸起
    \item 内弯:工件凹陷
\end{enumerate}

测量细丝直径:
\begin{equation}
    \begin{aligned}
           & \tan{\theta}=\frac{\lambda}{2b}                            \\
        d= & tan{\theta{}}\times{}L=\frac{\lambda}{2}\cdot{}\frac{L}{b}
    \end{aligned}
\end{equation}

\subsection{牛顿环}

由一块平板玻璃和一平凸透镜组成。

由于 $e$ 变化呈曲线,条纹距离变化不等,中疏边密。

\begin{equation}
    \delta=2e+\frac{\lambda}{2}\\
\end{equation}

\begin{equation}
    r=\sqrt{2eR}=\sqrt{(\delta-\frac{\lambda}{2})R}=
    \left\{
    \begin{aligned}
         & \sqrt{(k-\frac{1}{2})R\lambda} & \mbox{明环半径} \\
         & \sqrt{kR\lambda}               & \mbox{暗环半径}
    \end{aligned}
    \right.
\end{equation}

\subsubsection{牛顿环应用}

测量透镜的曲率半径:
\begin{equation}
    r_k^2=kR\lambda\\
    r_{k+m}^2=(k+m)R\lambda\\
    R=\frac{r_{k+m}^2-r_k^2}{m\lambda}
\end{equation}

工件标准件对比。

\section{迈克耳孙干涉仪}

\begin{equation}
    \Delta{d}=\Delta{k}\frac{\lambda}{2}
\end{equation}

在 $M_2$ 反射镜的左侧插入介质后:

光程差变化:$\delta{}'=2d+2(n-1)t$。

介质片厚度:$t=\frac{\Delta{k}}{n-1}\cdot\frac{\lambda}{2}$

\section{光的衍射}

光在传播过程中碰到\textbf{尺寸比光的波长大得不多}的障碍物时,光会传播到障碍物的阴影区并形成明暗变化的光强分布的现象。

衍射后会形成明暗相间的图样,中央明纹最亮,两侧显著递减。

\begin{enumerate}
    \item 单缝夫琅禾费衍射
    \item 圆孔夫琅禾费衍射
    \item 矩形孔夫琅禾费衍射
    \item 长方孔夫琅禾费衍射
\end{enumerate}

\subsection{菲涅耳衍射}

光源-障碍物-接收屏距离\textbf{有限远}。

\subsection{夫琅禾费衍射}

光源-障碍物-接收屏距离\textbf{无限远}。

\subsubsection{明暗条件}

衍射角$\theta$:衍射光线与单缝平面法线间的夹角(边缘衍射光线穿过透镜中心),向上为$+$ ,反之为负,取值范围:$0\to\frac{\pi}{2}$ 。

衍射角相同,汇聚在焦平面同一点,光强由这些平行光线干涉结果决定。

中央明条纹:$\theta{}=0$。

边缘最大光程差:$\delta{}=a\sin{\theta}$。

\subsubsection{半波带法}

\begin{equation}
    \begin{aligned}
        a\sin{\theta} & =\pm{}(2k+1)\frac{\lambda{}}{2}    & (k=1,2,3,\dots)~~ & \mbox{剩余半个半波带发光}   \\
        a\sin{\theta} & =\pm{}2k\cdot{}\frac{\lambda{}}{2} & (k=1,2,3,\dots)~~ & \mbox{两个半波带抵消,暗纹}
    \end{aligned}
\end{equation}

与干涉明暗纹条件\textbf{相反}!

两个半波带光程差为$\frac{\lambda}{2}$,两条光线\textbf{干涉相消}。

\subsubsection{中央明条纹角宽度和线宽度}

中央明纹角宽度:

\begin{equation}
    \Delta{\varphi_0}=2\varphi_1\approx{}2\frac{\lambda}{a}
\end{equation}

中央明纹线宽度:

\begin{equation}
    \Delta{x_0}=2f\cdot\tan{\varphi_1}\approx{}2f\varphi_1=2f\frac{\lambda}{a}
\end{equation}

其他明纹线宽度:

\begin{equation}
    \Delta=f\cdot\tan{\varphi_1}\approx{}f\varphi_1=f\frac{\lambda}{a}
\end{equation}

中央明纹宽度时其他两倍。

\subsection{圆孔衍射}

一级暗纹的衍射角:

\begin{equation}
    \sin{\theta_1}=1.22\frac{\lambda}{D}
\end{equation}

D 为圆孔直径。


两个点光源衍射满足瑞利法则:一个点光源的衍射图样的主极大正好与另一光源的第一极小重合时,恰能分辨两个光源。

最小分辨角:$\theta_R=\theta_0\approx{}\sin{\theta_1}=1.22\frac{\lambda}{D}$

提高光学仪器的分辨本领:$D$ 变大,$\lambda$ 减小。

\subsection{惠更斯-菲涅耳原理}

惠更斯原理:波阵面上的每一点都可以看作发射子波的新波源,子波的包络面就是该时刻的波阵面。

\subsubsection{子波的干涉}

次级子波相干叠加。

\section{光栅衍射}

光栅:等宽等距的狭缝排列构成的光学元件,适合测量单色光波长。

光栅常数:$d=a+b=\frac{1}{N}$,量级为$10^{-5}\to10^{-6}\mathrm{m}$。

明纹条件:$(a+b)\sin{\theta}=\pm{}k\lambda$。

暗纹条件:叠加后矢量为零。

\begin{enumerate}
    \item 亮度调制:光强受到单缝衍射影响,主极大光强分布不均匀。
    \item 缺级:满足单缝暗纹条件的光栅主极大缺级\\(干涉极大衍射极小)。
          \begin{equation}
              a\sin{\theta}=\pm{}k\lambda\\
              d\sin{\theta}=\pm{}k\lambda\\
              k=\frac{d}{a} \mbox{时消失}
          \end{equation}
\end{enumerate}

\subsection{衍射光谱}

\begin{enumerate}
    \item $\theta{}_{k+1}(purple)>\theta{}_{k}(red)$ 不重叠
    \item $\theta{}_{k+1}(purple)=\theta{}_{k}(red)$ 恰好不重叠
    \item $\theta{}_{k+1}(purple)>\theta{}_{k}(red)$ 重叠
\end{enumerate}

\section{光的偏振}

光的偏振现象:光是\textbf{横波}。

自然光:在各个方向的光矢量都存在,且相等,正交分解后得到互相垂直的光振动。

马吕斯定律(计算偏振片获得的线偏振光的光强):
\begin{equation}
    I=I_0\cos^2{\alpha}\\
    \alpha\mbox{为光振动方向和通光方向的夹角}
\end{equation}

折射定律:
\begin{equation}
    \frac{\sin{i}}{\sin{\gamma}}=\frac{n_2}{n_1}\\
    i+\gamma=\frac{\pi}{2}
\end{equation}

布儒斯特定律($i_0$ 为起偏角、布儒斯特角):
\begin{equation}
    \tan{i_0}=\frac{n_2}{n_1}\\
    \mbox{则反射光为完全偏振光,其中反射光与折射光垂直}
\end{equation}

\subsection{偏振器、起偏器和检偏器}

二向色性材料:吸收一方向的光,只让与该方向垂直的光通过的材料。

偏振片:涂有二向色性材料的透明薄片。

偏振化方向:只让特定光通过的方向。

起偏:使自然光变成线偏振光。

检偏:检查入射光的偏振性。

起偏器(偏振片)同时可以作为检偏器。

临界角:光由光密介质射入光疏介质中,折射角小于入射角,而如果满足 $\alpha=\arcsin{\frac{n_2}{n_1}}$ ,则折射角为 $90^{\circ}$。

\newpage