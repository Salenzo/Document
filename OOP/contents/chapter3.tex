\chapter{继承}
\newpage

\section{访问控制}

\begin{table}[!htbp]
    \centering
    \begin{tabular}{lccc}
        \toprule
                   & private(default)        & protected & public \\
        \midrule
        私有派生   & \multirow{3}*{不可访问} & 私有      & 私有   \\
        受保护派生 &                         & 受保护    & 受保护 \\
        公有派生   &                         & 受保护    & 公有   \\
        \bottomrule
    \end{tabular}
\end{table}

\subsection{改变访问限制}

使用 using 声明改变访问限制:
\begin{lstlisting}[frame=shadowbox]
    class D : B {
    private:
        float y;
        using B::set_x(); //set_x() is public in B
    }
\end{lstlisting}

\subsection{同名覆盖}

又称名字隐藏,如果派生类中添加的数据成员和基类中的数据成员(函数)同名,\textbf{派生类中的优先},并隐藏了继承来的数组成员(函数),且可继承。

\subsection{间接继承}

数据成员可以通过继承链路继承,继承链路包括了基类和所有派生类。
继承可以通过直接基类派生,也可以从间接基类派生。

\newpage

\section{继承机制下的构造和析构函数}

\begin{table}[!htbp]
    \centering
    \begin{tabular}{lccc}
        \hline
        构造函数 & 基类   & $->$ & 派生类 \\
        析构函数 & 派生类 & $->$ & 基类   \\
        \hline
    \end{tabular}
\end{table}

\subsection{继承机制下的构造函数}

创建派生对象时,基类的构造函数将被自动调用,对\textbf{派生类对象中的基类部分}初始化。
而派生类自己的构造函数则负责对象中的\textbf{派生类部分}的初始化。

默认构造函数:
\begin{enumerate}
    \item 没有带明显形参的构造函数
    \item 提供了默认实参的构造函数
\end{enumerate}

创建派生类对象时,必须显式(带括号)或隐式(无括号)调用。
如果基类有构造函数但没有默认构造函数,派生类的构造函数\textbf{必须}显式调用基类的某个构造函数。

\subsection{继承机制下的析构函数}

执行次序与构造函数相反,从派生类到基类。

\section{多继承}

派生类有多个基类,是所有基类的组合体。

\subsection{虚基类}

为解决从同一间接基类继承多次而造成浪费和混淆的问题,使用虚基类继承,从而仅将同名数据成员的一份拷贝发送到派生类。

\section{继承方式}

\subsection{保护继承}

\begin{enumerate}
    \item 基类中所有公有成员在派生类中是保护成员
    \item 基类中所有私有成员在派生类中是保护成员
    \item 基类中所有私有成员仅在基类中可见
\end{enumerate}


\subsection{私有继承}

\begin{enumerate}
    \item 基类中所有公有成员在派生类中是私有成员
    \item 基类中所有私有成员在派生类中是私有成员
    \item 基类中所有私有成员仅在基类中可见
\end{enumerate}

\newpage